\documentclass[a4paper, 11pt]{article}
\usepackage{comment} % enables the use of multi-line comments (\ifx \fi) 
\usepackage{lipsum} %This package just generates Lorem Ipsum filler text. 
\usepackage{fullpage} % changes the margin
\usepackage{amsmath}
\begin{document}
%Header-Make sure you update this information!!!!
\noindent
\large\textbf{Group 19 project 1st draft} \hfill \textbf{Group memebers} \\
\normalsize Stats 506 \hfill  Mimi Tran, Zhenbang Jiao, Yijia Zhang \\
Date: Nov/26 2018

\section{Topic}
Topic: Linear regression
\subsection{Introduction}
Formal definition from \emph{Linear Models with R}: Regression analysis is used for explaining or modeling the relationship between a single variable Y, called the response, output or dependent variable; and one or more predictor, input, independent or explanatory variables, $X_1$...$X_p$. When $p=1$, it is called simple regression but when $p>1$ it is called multiple regression or sometimes multivariate regression. (Faraway, J 2009)\newline
An informal interpretation is that linear regression establish a linear relationship between the response variable and independent variable(s) in a data set. It is widely applied in many areas such as sciences, engineering and machine learning. \newline
Mathematically:
\begin{equation}
Y = \beta_0 + \beta_1 X_1 + \beta_2 X_2 + ... + \beta_n X_n
\end{equation}
\subsection{Solving linear regression problems with least square}
One well defined way to solve for $\beta$ is called the least square method: \newline
Using above definition, we can write:\begin{align*}
y = X \beta + \epsilon
\end{align*} 
with $y = [y_1, ..., y_n]^T$, $\epsilon = [\epsilon_1,..., \epsilon_n]^T$, $\beta = [\beta_0,...,\beta_m]^T$ and:\begin{equation}
    X = \begin{bmatrix}
    1 & x_{11}& x_{12}  & \dots & x_{1m} \\
    1 & x_{21}& x_{22}  & \dots & x_{2m} \\
    \vdots & \vdots & \vdots & \ddots & \vdots \\
    1 & x_{n1}& x_{d2}  & \dots & x_{nm}
\end{bmatrix}
\end{equation}
In above expression, $\epsilon$ is the residual, and one way to obtain best estimation of $\beta$ is to minimize:\begin{equation}
    \sum_{i=1}^n \epsilon_i^2 = \epsilon^T\epsilon
\end{equation}
The above equation can be rewritten as: \begin{equation}
    (y - X\beta)^T (y - X\beta) 
\end{equation}
Now, we take the differential of (4) with respect to $\beta$, notice that to minimize (4), the graident should be zero; thus: \begin{equation}
    X^TX\hat{\beta} = X^T y 
\end{equation}
in which $\hat{\beta}$ is our estimation of $\beta$. Rearranging (5) we get: \begin{equation}
     \hat{\beta} = (X^TX)^{-1}X^T y
\end{equation}
This is the norm equation. 
\subsection{example with data}
Need to fill
\section{Diagnostic and Model Selection}
Normally, there are numerous variables in a data set, and some of variables are actually correlated; then one problem is how to establish a rather simple model which could be easily interpret. One popular technique is the \emph{criterion-based Model Selection}. The idea is to choose model with respect to a specific criterion that measures the behavior of fit. In our project, we will introduce three common criterion: the \emph{Akaike information criterion (AIC)}, the \emph{Bayes information criterion (BIC)}, and the \emph{adjusted R square}. 
\subsection{AIC}
Before defining AIC, we first define the following values: \newline
Number of independent predictors p: $p = \# X_i$ \newline
Residual sum of square - RSS: \begin{equation}
    RSS = \hat{\epsilon}^T\hat{\epsilon} = (y - X\hat\beta)^T(y - X\hat\beta)
\end{equation}
And
\begin{equation}
    AIC = n \ln{(RSS/n)} + 2(p+1)
\end{equation}
Pick the model minimizes AIC.
\subsection{BIC}
\begin{equation}
    BIC = n \ln{(RSS/n)} + (p+1)\ln n
\end{equation}
Pick the model minimizes BIC.
\subsection{Adjusted R square}
The R square value is defined as:
\begin{equation}
    R^2 = 1 - \frac{RSS}{TSS}
\end{equation}
The adjusted R square value is:
\begin{equation}
    1 - \frac{n-1}{n-p-1}(1 - R^2)
\end{equation}
Select a model that maximize the adjusted $R^2$ value 
\subsection{Examples}
Fill

\begin{thebibliography}{9}

\end{thebibliography}
To be fill in
\end{document}
